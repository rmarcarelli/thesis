\chapter{Concluding Remarks}

It is difficult to overstate the success that particle physics has enjoyed over the past century. Since 1925, our catalog of fundamental particles has grown from two (the electron and photon) to thirty (including antiparticles).\footnote{There is some ambiguity as to what constitutes a particle. I have counted each {\it Dirac} fermion field as two particles (a particle and antiparticle), but this is tricky for the neutrino which is a {\it Weyl} fermion in the current formulation of the Standard Model. I still count each neutrino and anti-neutrino as a separate particle. I have also counted the gluon as {\it one} particle instead of {\it eight}, treating it as a single field which is in the adjoint representation of $SU(3)_C$ and hence has eight colored components. While this is consistent with treating each quark as {\it one} particle as opposed to {\it three} (one for each color in the $SU(3)_C$ triplet), it is somewhat at odds with our treatment of the gauge bosons in $SU(2)_L$. However, given that $SU(2)_L$ is broken by the Higgs VEV and a spectrum of different massive states arise, I believe this number is internally consistent. One can more objectively refer to the number of {\it particle degrees of freedom} between 1925 and today (four and 128), but this is less narratively compelling.} In that time, the framework of quantum field theory was developed to describe the nature of these particles, and the framework of local symmetries (or gauge theories) to describe their interactions. The puzzle of missing energy in nuclear decays led to the discovery of the neutrino and the formulation of the weak interaction. Deep inelastic scattering experiments and the tower of hadrons observed in cosmic rays revealed the existence of quarks and laid the groundwork for the development of the strong interaction. The observation of CP-violation in kaon decays and discovery of the $\tau$ lepton pointed to the existence of three generations of each fermion type. The need to give mass to the weak interaction's gauge bosons while preserving gauge symmetry led to the development of the Brout-Englert-Higgs mechanism and the prediction of the Higgs boson, which also provided a natural framework for fermion mass generation. This is but a brief account of the milestones that culminated in the development of the Standard Model of particle physics, all in the span of a human lifetime. 

Despite this success, we are currently in a drought of new discoveries. The Standard Model as we know it today was finalized in the late 1970s with the unification of the electromagnetic and weak interactions into the electroweak theory, and the establishment of quantum chromodynamics as the theory of the strong force. In the years that followed, its predictions were slowly verified: in the early 1980s, with the discovery of the $W^\pm$ and $Z$ bosons; in 1995, with the discovery of the top quark; and finally, in 2012, with the discovery of the Higgs boson. In that period and up until today, no new elementary particles or forces have been observed. The Standard Model continues to agree with experimental data across a vast range of energies and processes, and remains to this day our most accurate description of fundamental physics ever constructed.


And yet, we are reasonably certain that the Standard Model is an incomplete description of the particle content of our Universe. Our astrophysical observations indicate that 85\% of the matter in the universe is not comprised of Standard Model particles, but is instead {\it dark} matter that has very weak Standard Model interactions (if any at all).\footnote{While ``Modified Newtonian Dynamics'' (MOND) can also solve this problem by changing the form of the gravitational interaction at large scales, observations of gravitational lensing from dark matter in the empty space around colliding galaxies \cite{Clowe:2006eq} and possibly even galaxies which contain no dark matter at all \cite{van_Dokkum_2018} render a particle solution more likely.} In addition, the observation of neutrino oscillations is a direct hint that particles beyond the Standard Model must exist. Perhaps most strikingly, the Standard Model is completely silent on the force we are most familiar with: gravity. Nearly every attempt to unify gravity with the other three forces inevitably predicts a vast spectrum of particles, none of which have appeared in any of our collider experiments. 

Even within the Standard Model, there are both inconsistencies and mysteries. Ignoring neutrinos, the masses of the fermions still span four orders of magnitude, and the Standard Model offers no explanation for this hierarchy. Worse still, the Standard Model predicts that quantum corrections to the Higgs boson mass should push it toward the Planck scale, and yet we observe it to have a mass of only $125~{\rm GeV}$. In addition, the CP-violating gluon-gluon interaction $\tfrac{\theta}{4\pi^2} {\rm Tr}\{{G}_{\mu\nu}\tilde{G}^{\mu\nu}\}$ is apparently absent from the Standard Model, despite respecting all of the Standard Model gauge symmetries.

A variety of particle explanations for the phenomena described above were reviewed in Chapter \ref{fv_qft}. They almost all have one feature in common that is completely absent in the Standard Model: charged lepton flavor violation (CLFV). While the degree of this flavor-violation can be made arbitrarily small by tuning the couplings in each model, we have the benefit of knowing from neutrino oscillations that lepton flavor is {\it not} a good symmetry of the Universe.  Then it is not a matter of {\it if} CLFV processes occurs, but rather {\it how often}. If the only process which directly contributes to CLFV is neutrino oscillation, we will likely never observe a CLFV event in our lifetimes. However, if any of the models discussed in Chapter \ref{fv_qft} is realized in our Universe, CLFV processes could have rates many orders of magnitude higher than neutrino oscillation-mediated processes.

Given the ubiquity of CLFV in extensions to the Standard Model, it is a very promising avenue for discovery. In Chapter~\ref{lfv}, we reviewed some of the leading constraints on CLFV couplings: limits on the branching fraction of the LFV lepton decays $\ell_i \rightarrow \ell_j \ell_k\bar{\ell}_l$ and $\ell_i \rightarrow \ell_j \gamma$. While the resulting constraints are strong, they are dependent on products of the CLFV couplings, and hence sensitive to relative hierarchies between said couplings. In addition, the degree of PV in the interaction can lead to a suppression of the limits for certain channels, in particular for purely chiral interactions. Hence, when placing limits on a new CLFV model of physics, one must keep these properties of the model in mind to ensure that the results are accurate. 

In order to find exclusions on {\it singular} couplings as opposed to coupling products, we turned to the lepton electric and magnetic dipole moments. The strongest constraints on singular couplings (which in some instances out-perform the LFV lepton decay constraints above) come from the electron EDM on $|g_{e\ell}|$. However, these constraints are only present for CP-violating interactions, so neither purely parity-conserving nor purely chiral interactions are constrained by this measurement. On the other hand, interactions with any degree of PV contribute to the magnetic dipole moments of the leptons. Limits on these couplings assuming their contributions lie within the current experimental uncertainty were obtained in Section~\ref{sec:mdm_constraints}; it was found once again that in some cases, there is a vast suppression in the contribution from chiral interactions compared to purely parity conserving interactions, resulting in weaker limits for the chiral scenario. Of course, both of these measurements are currently (at least somewhat) plagued by anomalies, so agreement must be found between the theoretical and experimental values before the bounds can be taken at face value. In the interim, we have presented flavor-violating solutions to the electron and muon $g-2$ anomalies assuming a singular flavor off-diagonal coupling $g_{\ell\tau}$, exploring how the result varies with the PV angle $\theta_{\ell\tau}$.

The work in Chapter~\ref{lfv} only explored scalars which contribute to flavor-changing neutral currents. It would be interesting to see whether a similar chiral suppression can be obtained for vectors. Results from Ref.~\cite{Zhevlakov:2023jzt} indicate that this is indeed the case for LFV dark photon contributions to the electron $g-2$, although more work needs to be done to determine whether this generalizes. In addition, charged scalars and vectors can also contribute to the LFV decays and dipole moments discussed above; it would be interesting to explore the role that PV plays in these contexts.

In Chapter \ref{upc}, we explored an alternative process that has the potential to probe LFV couplings which is largely insensitive both to model hierarchies and degree of PV in the interaction (at least for heavy particles): production of a new boson in lepton-nucleus collisions via the process $\ell^- A_Z \rightarrow \ell'^- A_Z \varphi$. For concreteness, we have compared four experimental set-ups: the E137 beam dump experiment at SLAC, the upcoming EIC at BNL, a hypothetical $1~{\rm TeV}$ Muon Beam Dump (dubbed MuBeD), and a hypothetical $1~{\rm TeV}$ Muon (Synchrotron)-Ion Collider (dubbed MuSIC). Due to the additional luminosity from a solid target combined with the large pseudo-rapidity reach, we find that beam dump experiments are a superior mode for production and detection of {\it light} bosons ($m_\varphi \lesssim 1~{\rm GeV}$), but lepton-ion colliders become competitive for heavier bosons. Improvements in luminosity or pseudo-rapidity reach for lepton-ion colliders would improve their sensitivity. Of course, there are also other possible production modes that may be more promising at these experiments than their beam-dump counterparts.

The code developed for this chapter allows for a detailed reconstruction not only of the production cross-section of the new boson, but its kinematical distributions over a wide range of masses as well. These methods should be broadly applicable to any $2\rightarrow 3$ scattering processes, provided integrability of the amplitude over $\phi_q$. Alternatively, in the event that the integral over $\phi_q$ is intractable, one can still use the identification (\ref{eq:WW_sub}) in situations for which the Weizs\"acker-Williams approximation applies. In particular, we find that while the {\it Improved} Weizs\"acker-Williams approximation can differ somewhat substantially from the true cross-section at heavy boson masses, the original Weizs\"acker-Williams approximation remains reliable across all of the processes considered, and can save on computational costs. To conclude the chapter, we explored limits one can obtain from an LFV scalar at the EIC, MuBeD, and MuSIC, finding that limits on $g_{\ell\tau}$ are not as competitive as those considered in Chapter~{\ref{lfv}} {\it in general}, but have the benefit of being mostly independent of the other model parameters (up to the branching fraction of the final-state boson).


In Chapter \ref{alp_collider}, we examined collider constraints on a certain class of CLFV particles: GeV-scale leptophilic ALPs. Such particles are realized e.g. in ${\Bbb Z}_N$ Froggatt-Nielsen models \cite{Greljo:2024evt} and the composite dark sector scenario in Ref.~\cite{Davoudiasl:2017zws}. We began by exploring the ability to produce and detect these particles at CERN in the event that they have a substantial coupling to the Higgs. We find that for substantial Higgs couplings ($C_{ah}/\Lambda^2 \sim 1~{\rm TeV}^{-2}$, $C_{ah}'/\Lambda^2 \sim 10~{\rm TeV}^{-2}$), the LFV coupling $C_{\tau \ell}/\Lambda$ has already been probed down to $10^{-5}~{\rm TeV}^{-1}$ at CMS and $10^{-6}~{\rm TeV}^{-1}$ at ATLAS, far exceeding the current limits from LFV lepton decays. In addition, it was found that the proposed MATHUSLA experiment would be able to push the coupling limits even further to $C_{\tau \ell}/\Lambda \gtrsim 10^{-8}~{\rm TeV}^{-1}$ at the high-luminosity LHC.

However, in the event that the Higgs couplings are small ($C_{ah}/\Lambda^2 \lesssim 1~{\rm TeV}^{-2}$, $C_{ah}'/\Lambda^2 \lesssim 10~{\rm TeV}^{-2} $), the limits and projections weaken and eventually disappear. In this event, we apply the results of Chapter~\ref{upc} to find model-independent limits. As with scalars, it is found that these limits are substantially weaker, probing couplings at around $C_{\tau\ell}/\Lambda \gtrsim 1~{\rm TeV}^{-1}$, but represent absolute limits in the absence of other model couplings. For a muon beam dump, however, we find that the most optimistic scenario (for which a $1~{\rm TeV}$ muon beam is incident on a $2~{\rm m}$ block of lead for ${\cal O}(1~{\rm year})$ of total operation, corresponding to $10^{20}$ muons on target), the resulting limits are competitive with existing limits from LFV lepton decays, while remaining largely insensitive to the other model parameters.

Finally, in Chapter~\ref{bosons}, we explored the possibility of detecting displaced decays of hidden gauge bosons produced via the process discussed in Chapter~\ref{upc} at the EIC, MuBeD, and MuSIC. A certain class of these bosons, the $U(1)_{L_i - L_j}$ gauge bosons, exhibit CLFV couplings after spontaneous breaking of the abelian symmetry, but here we only focus on production via the on-diagonal channel $\ell A_Z \rightarrow \ell A_Z A'$. The EIC is able to fill in a currently untouched region of the parameter space near $(m_{A'}, g_{A'}) \sim (100~{\rm MeV}, 10^{-5})$ for a dark photon, $U(1)_{B-L}$, $U(1)_{L_e - L_\mu}$, and $U(1)_{L_e - L_\tau}$. MuSIC and MuBeD are sensitive to all of these except $U(1)_{L_e - L_\tau}$, but are additionally sensitive to $U(1)_{L_\mu - L_\tau}$. Surprisingly, we find that a similar analysis at MuSIC yields slightly worse results than at the EIC, despite the vast increase in available energy at the hypothetical collider. This is mostly due to the increased pseudorapidity of the produced bosons at MuSIC, but is also due to a slight decrease in the production cross-section at MuSIC compared to the EIC in the relevant mass range. Given its higher luminosity and pseudorapidity reach, MuBeD would be able to probe much heavier dark bosons with displaced decays, providing limits or possibly detecting particles near $(m_{A'}, g_{A'}) \sim (1~{\rm GeV}, 10^{-7})$, a region which is entirely untouched by modern experiments.  

This chapter focused solely on parity-conserving couplings of the dark boson, so it would be interesting to see whether stronger limits can be obtained for axial vectors (whose production cross-section per Fig.~\ref{fig:production_crossx} appears to be substantially larger for some masses). However, the mass range for which the PV nature of the interaction matters is $m_{A'} < m_e$ at the EIC and $m_{A'} < m_\mu$ at MuBeD and MuSIC, which are regions that are already heavily covered by existing experiments. However, in the event that such a signal is found, one could potentially use the left-right asymmetry with a polarized beam to determine the PV nature of the interaction. Finally, we note that we have only considered production off of the lepton side off of the interaction, whereas dark photons and $U(1)_{B-L}$ gauge bosons will also have significant interactions with the nucleus. These diagrams are suppressed by an evaluation of the form-factor at the mass-squared of the boson, $G(m_{A'}^2)$. While this may matter for $m_{A'} \lesssim 10~{\rm MeV}$, it drops off rapidly for larger masses due to the expression for the form factor (\ref{eq:nuc_ff_coh}).
